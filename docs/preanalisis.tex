% Options for packages loaded elsewhere
\PassOptionsToPackage{unicode}{hyperref}
\PassOptionsToPackage{hyphens}{url}
\PassOptionsToPackage{dvipsnames,svgnames*,x11names*}{xcolor}
%
\documentclass[
]{article}
\usepackage{lmodern}
\usepackage{setspace}
\usepackage{amssymb,amsmath}
\usepackage{ifxetex,ifluatex}
\ifnum 0\ifxetex 1\fi\ifluatex 1\fi=0 % if pdftex
  \usepackage[T1]{fontenc}
  \usepackage[utf8]{inputenc}
  \usepackage{textcomp} % provide euro and other symbols
\else % if luatex or xetex
  \usepackage{unicode-math}
  \defaultfontfeatures{Scale=MatchLowercase}
  \defaultfontfeatures[\rmfamily]{Ligatures=TeX,Scale=1}
\fi
% Use upquote if available, for straight quotes in verbatim environments
\IfFileExists{upquote.sty}{\usepackage{upquote}}{}
\IfFileExists{microtype.sty}{% use microtype if available
  \usepackage[]{microtype}
  \UseMicrotypeSet[protrusion]{basicmath} % disable protrusion for tt fonts
}{}
\makeatletter
\@ifundefined{KOMAClassName}{% if non-KOMA class
  \IfFileExists{parskip.sty}{%
    \usepackage{parskip}
  }{% else
    \setlength{\parindent}{0pt}
    \setlength{\parskip}{6pt plus 2pt minus 1pt}}
}{% if KOMA class
  \KOMAoptions{parskip=half}}
\makeatother
\usepackage{xcolor}
\IfFileExists{xurl.sty}{\usepackage{xurl}}{} % add URL line breaks if available
\IfFileExists{bookmark.sty}{\usepackage{bookmark}}{\usepackage{hyperref}}
\hypersetup{
  pdftitle={Pre-registration},
  colorlinks=true,
  linkcolor=blue,
  filecolor=Maroon,
  citecolor=Blue,
  urlcolor=Blue,
  pdfcreator={LaTeX via pandoc}}
\urlstyle{same} % disable monospaced font for URLs
\usepackage[margin=0.78in]{geometry}
\usepackage{longtable,booktabs}
% Correct order of tables after \paragraph or \subparagraph
\usepackage{etoolbox}
\makeatletter
\patchcmd\longtable{\par}{\if@noskipsec\mbox{}\fi\par}{}{}
\makeatother
% Allow footnotes in longtable head/foot
\IfFileExists{footnotehyper.sty}{\usepackage{footnotehyper}}{\usepackage{footnote}}
\makesavenoteenv{longtable}
\setlength{\emergencystretch}{3em} % prevent overfull lines
\providecommand{\tightlist}{%
  \setlength{\itemsep}{0pt}\setlength{\parskip}{0pt}}
\setcounter{secnumdepth}{5}
\usepackage{caption}
\captionsetup[figure, table]{labelfont={bf},labelformat={default},labelsep=period}
\usepackage{graphicx}
\usepackage{float}
\usepackage{booktabs}
\usepackage{longtable}
\usepackage{array}
\usepackage{multirow}
\usepackage{wrapfig}
\usepackage{float}
\usepackage{colortbl}
\usepackage{pdflscape}
\usepackage{tabu}
\usepackage{threeparttable}

\title{Pre-registration}
\author{}
\date{\vspace{-2.5em}}

\begin{document}
\maketitle

{
\hypersetup{linkcolor=}
\setcounter{tocdepth}{2}
\tableofcontents
}
\setstretch{1.5}
\hypertarget{perceptions-and-preferences-for-meritocracy-scale-ppm-s}{%
\section{Perceptions and Preferences for Meritocracy Scale
(PPM-S)}\label{perceptions-and-preferences-for-meritocracy-scale-ppm-s}}

(link to \href{docs/preanalisis.pdf}{PDF file})

\hypertarget{research-questions}{%
\section{Research questions}\label{research-questions}}

Is it possible to identify and measure perceptions and preferences for
meritocracy as two distinct but related factors?

\hypertarget{general-description-and-hypotheses}{%
\section{General description and
hypotheses}\label{general-description-and-hypotheses}}

This research builds upon previous studies attempting to develop a
comprehensive approach to measuring meritocratic beliefs (Duru-Bellat \&
Tenret, \protect\hyperlink{ref-Duru-bellat2012}{2012}; Kunovich \&
Slomczynski, \protect\hyperlink{ref-Kunovich2007}{2007}; Newman et al.,
\protect\hyperlink{ref-Newman2015}{2015}; Reynolds \& Xian,
\protect\hyperlink{ref-Reynolds2014}{2014}). The proposed measurement
model is based on two axis of analysis, as depicted in Table 1. The
first one distinguishes between different types of ``beliefs'', using
instead the terms perceptions and preferences for meritocracy.
Perceptions refers to the extent to which people see meritocracy working
in their society, which in terms of measurement relates to the ``reasons
to get ahead'' battery, whereas preferences refer to normative
expectations that are usually linked to a ``should'' expression
(e.g.~whether hard work should be related to payment). The second axis
consider the distinction between meritocratic and non-meritocratic
dimensions. This aspect has been usually treated as different ends of a
same continuum in part of the previous research, an assumption that
requieres empirical scrutiny. These non-meritocratic elements usually
refer to the use of personal contacts or family advantages to get ahead
in life.

Table 1: Model of perceptions and preferences for meritocracy and
non-meritocracy

\begin{longtable}[]{@{}lll@{}}
\toprule
& Perceptions & Preferences\tabularnewline
\midrule
\endhead
\emph{Meritocracy} & &\tabularnewline
\emph{Non-meritocracy} & &\tabularnewline
\bottomrule
\end{longtable}

\textbf{Hypotheses}

\begin{itemize}
\tightlist
\item
  H1. The perception of meritocracy is a latent variable based on
  indicators of the importance attributed to talent and the effort to
  get ahead in life.
\item
  H2. The non-meritocratic perception is a latent variable that derives
  from two indicators related to the agreement with the statement that
  people with contacts and rich parents manage to get ahead.
\item
  H3. Meritocratic preferences behave as the latent variable based on a
  normative value of effort and talent.
\item
  H4. Non-meritocratic preferences behave as a latent variable based on
  the normative value of the use of personal contacts and having rich
  parents.
\end{itemize}

\hypertarget{type-of-study}{%
\section{Type of study}\label{type-of-study}}

The scale will be included in an online three-wave panel survey study.
This study covers among other things an experiment of the effect of
information about poverty and inequality on opportunity beliefs. The
scale will be presented in wave 1 and wave 3.

\hypertarget{sample}{%
\section{Sample}\label{sample}}

\hypertarget{detailed-data-and-explanations-of-its-use.}{%
\subsection{Detailed data and explanations of its
use.}\label{detailed-data-and-explanations-of-its-use.}}

The data will come from a representative sample of the large cities of
Chile (Gran Santiago, Valparaíso-Viña del Mar, Concepción and
Antofagasta). In order to ensure representativeness, quotas for sex,
education and age will be provided to the agency in charge of the online
fieldwork.

\hypertarget{magnitude-of-the-sample-and-explanation-of-the-magnitude.}{%
\subsection{Magnitude of the sample and explanation of the
magnitude.}\label{magnitude-of-the-sample-and-explanation-of-the-magnitude.}}

The sample will have an initial number for the first wave of 2100 cases,
which are expected to be at least 1800 in the second wave. From the
second to the third wave it has been considered as ``free fall'', from
which we expect to retain about 1500 cases.

The sample size was calculated from power analysis that used information
from a pilot study that was applied in June 2019. This pilot study
consisted of an online experiment applied to 949 subjects, randomly
separating participants to two stimuli and a placebo, leaving a third of
the sample in each condition. From these data, the low power of a
two-tailed test was calculated, considering an average treatment effect
of 0.26, with a standard deviation of the result of the treatment group
of 1.23 and a significance level of 0.05. Based on multiple comparison
correlations (Bonferroni), we adjusted alpha for six tests (three
results * two comparisons between conditions and placebo). The results
indicated that the number of subjects exposed to each stimulus should be
580, so we will work with three samples of 600 cases, which gives us a
total of 1800 cases for the second wave in which the experiment will be
applied. To guarantee the above and in consideration of a high
non-response rate, we will start with a sample of 2100 subjects.

\hypertarget{data-collection-procedure}{%
\subsubsection{Data collection
procedure:}\label{data-collection-procedure}}

Respondents will be invited to participate in the online survey under
the incentive provided by the external company. Respondents have 4
business days to answer the survey, with a maximum of 20 minutes per
survey. In order to achieve greater representativeness of the sample,
the quota method will be used, that is, the program will only allow
respondents with the required demographic characteristics. The quotas
used were age, sex and education. It should be noted that the time
between wave and wave is around 7 and 9 days.

\hypertarget{stop-rule}{%
\subsubsection{Stop Rule:}\label{stop-rule}}

The rule of detention is to reach the number of respondents indicated in
the quotas. However, there may be modifications depending on the
contingencies of the field work and the response rates.

\hypertarget{variables-to-use}{%
\section{Variables to use:}\label{variables-to-use}}

All the variables used in the scale use the following question and
answer options:

\begin{itemize}
\tightlist
\item
  To what extent do you agree or disagree with each of the following
  affirmations?

  \begin{itemize}
  \tightlist
  \item
    Response scale:

    \begin{itemize}
    \tightlist
    \item
      Strongly disagree
    \item
      Disagree
    \item
      Neither agree nor disagree
    \item
      Agree
    \item
      Totally agree
    \end{itemize}
  \end{itemize}
\end{itemize}

\hypertarget{variable-manipulation}{%
\subsubsection{Variable manipulation}\label{variable-manipulation}}

There will not be variable manipulation.

\hypertarget{indexes}{%
\subsubsection{Indexes}\label{indexes}}

The investigation of the validity of this scale does not rely on indices
but rather confirmatory factor analysis will be used to estimate the
value of the latent variables underlying the indicators.

\hypertarget{blinding-of-information}{%
\subsubsection{Blinding of information}\label{blinding-of-information}}

No special blinding is used for this study.

\hypertarget{randomization}{%
\subsubsection{Randomization}\label{randomization}}

To test the items' order effects we present the items in three different
order conditions which will be assigned randomly. The conditions are the
following:

\begin{itemize}
\tightlist
\item
  Order 1: the order in which items are presented in Table 1
\item
  Order 2: the items are ordered based on topics (work harder, talent,
  rich parents, connections), and for each of them they have to respond
  first about perceptions and then about preferences. For instance: a)
  those who work harder get greater rewards, b) those who work arder
  should get greater rewards \ldots{}
\item
  Order 3: the order is randomized within this condition
\end{itemize}

\hypertarget{analysis-plan}{%
\section{Analysis plan}\label{analysis-plan}}

To evaluate the hypotheses, confirmatory factor analysis (CFA) will be
used as we rely on a theory regarding the underlying four factors. The
method of estimation will be Weighted Least Squares Mean Variance
Approximation (WLMSV) for categorical ordered indicators. For the
analysis we will use the R package Lavaan (Rosseel,
\protect\hyperlink{ref-Rosseel2012}{2012}).

\hypertarget{inference-criteria}{%
\subsubsection{Inference criteria}\label{inference-criteria}}

The values that will be used as evaluation criteria for the goodness of
the adjustment of the model were taken from the proposal of Brown
(\protect\hyperlink{ref-Brown2008}{2008}) and are the following:

\begin{itemize}
\tightlist
\item
  Chi-square:\textgreater{} 0.05
\item
  Chi-square ratio:\textgreater{} 3
\item
  Comparative adjustment goodness index (CFI):\textgreater{} 0.95
\item
  Tucker-Lewis Index (TLI):\textgreater{} 0.95
\item
  Root of the average squared residual approximation \textless0.08.
\end{itemize}

\hypertarget{secondary-analysis}{%
\subsubsection{Secondary analysis}\label{secondary-analysis}}

To evaluate the metric stability of the measurement model (Davidov et
al., \protect\hyperlink{ref-Davidovetal2014}{2014}) we will conduct a
longitudinal invariance test using the first and third wave of our
study. Following Liu et al.
(\protect\hyperlink{ref-liuTestingMeasurementInvariance2017}{2017}), we
will test a series of four hierarchical models: Configural, Weak, Strong
and Strict invariance models for ordered-categorical indicators based on
the assumption that a five category likert scale cannot be treated as a
continous variable because can lead to biased parameter estimates.

\hypertarget{data-exclusion}{%
\subsubsection{Data Exclusion}\label{data-exclusion}}

All cases will be used as long as they do not show missing values in any
of the items on the scale. No imputation criteria will be used.

\hypertarget{ethics.}{%
\subsubsection{Ethics.}\label{ethics.}}

The experiment and survey instruments are approved by the IRB of the
University of Chile.

\hypertarget{bibliography}{%
\section*{Bibliography}\label{bibliography}}
\addcontentsline{toc}{section}{Bibliography}

\hypertarget{refs}{}
\leavevmode\hypertarget{ref-Brown2008}{}%
Brown, T. A. (2008). \emph{Confirmatory Factor Analysis for Applied
Research}. \url{https://doi.org/10.1198/tas.2008.s98}

\leavevmode\hypertarget{ref-Davidovetal2014}{}%
Davidov, E., Meuleman, B., Cieciuch, J., Schmidt, P., \& Billiet, J.
(2014). Measurement Equivalence in Cross-National Research. \emph{Annual
Review of Sociology}, \emph{40}(1), 55--75.
\url{https://doi.org/10.1146/annurev-soc-071913-043137}

\leavevmode\hypertarget{ref-Duru-bellat2012}{}%
Duru-Bellat, A. M., \& Tenret, E. (2012). Who's for Meritocracy?
Individual and Contextual Variations in the Faith. \emph{Comparative
Education Review}, \emph{56}(2), 223--247.

\leavevmode\hypertarget{ref-Kunovich2007}{}%
Kunovich, S., \& Slomczynski, K. M. (2007). Systems of distribution and
a sense of equity: A multilevel analysis of meritocratic attitudes in
post-industrial societies. \emph{European Sociological Review},
\emph{23}(5), 649--663. \url{https://doi.org/10.1093/esr/jcm026}

\leavevmode\hypertarget{ref-liuTestingMeasurementInvariance2017}{}%
Liu, Y., Millsap, R. E., West, S. G., Tein, J.-Y., Tanaka, R., \& Grimm,
K. J. (2017). Testing measurement invariance in longitudinal data with
ordered-categorical measures. \emph{Psychological Methods},
\emph{22}(3), 486--506. \url{https://doi.org/10.1037/met0000075}

\leavevmode\hypertarget{ref-Newman2015}{}%
Newman, B. J., Johnston, C. D., \& Lown, P. L. (2015). False
Consciousness or Class Awareness? Local Income Inequality, Personal
Economic Position, and Belief in American Meritocracy. \emph{American
Journal of Political Science}, \emph{59}(2), 326--340.
\url{https://doi.org/10.1111/ajps.12153}

\leavevmode\hypertarget{ref-Reynolds2014}{}%
Reynolds, J., \& Xian, H. (2014). Perceptions of meritocracy in the land
of opportunity. \emph{Research in Social Stratification and Mobility},
\emph{36}, 121--137. \url{https://doi.org/10.1016/j.rssm.2014.03.001}

\leavevmode\hypertarget{ref-Rosseel2012}{}%
Rosseel, Y. (2012). Lavaan: An R package for structural equation
modeling and more. Version 0.5--12 (BETA). \emph{Journal of Statistical
Software}, \emph{48}(2), 1--36.

\end{document}
